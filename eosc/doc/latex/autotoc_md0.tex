\label{_toc}%
 \subsection*{Table of contents}


\begin{DoxyItemize}
\item \href{#rationale}{\tt Rationale}
\begin{DoxyItemize}
\item \href{#richer}{\tt Richer A\+PI}
\begin{DoxyItemize}
\item \href{#richereos}{\tt E\+OS}
\item \href{#richertokenika}{\tt Tokenika}
\end{DoxyItemize}
\end{DoxyItemize}
\item \href{#building}{\tt Building}
\begin{DoxyItemize}
\item \href{#dependencies}{\tt Dependencies}
\item \href{#linux}{\tt Linux, Mac, ect.}
\item \href{#windows}{\tt Windows}
\end{DoxyItemize}
\end{DoxyItemize}

\label{_rationale}%
 \subsection*{\href{#toc}{\tt Rationale}}

For our work with eos small contracts, we have found that the original E\+OS {\ttfamily eosc} interface program is too much restrictive. First, it is hard to be used programmatically in a C++ code. Next, it is quite heavy as it is tightly connected to the whole of the E\+OS code. Also, it is not ready to be used in the Windows environment, while we plan to open Windows based contract development possibility.

It could be enough for us to develope a minimal C++ library, implementing the commands of the E\+OS {\ttfamily eosc}. However, it was a short step to to provide this library with an command line interface.

Finally, to make our work competitive to the original, and for fun, we have added a richer command option list. We dare to hope that this little work of ours could be included to the E\+OS project.

We already know how to use this richness\+: it is much ease to make a tool as tokenika \href{#}{\tt {\ttfamily eosc\+Bash}} that wraps the E\+OS {\ttfamily eosc} for bookkeeping.

\label{_richer}%
 \subsection*{\href{#toc}{\tt Richer A\+PI}}

\label{_richereos}%
 \#\#\# \href{#toc}{\tt E\+OS} 
\begin{DoxyCode}
./eosc get block -h
\end{DoxyCode}
 
\begin{DoxyCode}
ERROR: RequiredError: block
Retrieve a full block from the blockchain
Usage: ./eosc get block block

Positionals:
  block TEXT                  The number or ID of the block to retrieve
\end{DoxyCode}
 
\begin{DoxyCode}
./eosc get block 25
\end{DoxyCode}
 
\begin{DoxyCode}
\{
  "previous": "00000018b5e0ffcd3dfede45bc261e3a04de9f1f40386a69821780e063a41448",
  "timestamp": "2017-11-29T09:50:03",
  "transaction\_merkle\_root": "0000000000000000000000000000000000000000000000000000000000000000",
  "producer": "initf",
  "producer\_changes": [],
  "producer\_signature":
       "2005db1a193cc3597fdc3bd38a4375df2a9f9593390f9431f7a9b53701cd46a1b5418b9cd68edbdf2127d6ececc4d66b7a190e72a97ce9adfcc750ef0a770f5619",
  "cycles": [],
  "id": "000000190857c9fb43d62525bd29dc321003789c075de593ce7224bde7fc2284",
  "block\_num": 25,
  "refBlockPrefix": 623236675
\}
\end{DoxyCode}


\label{_richtokenika}%
 \subsubsection*{Tokenika}


\begin{DoxyCode}
./eosc get block -h
\end{DoxyCode}
 
\begin{DoxyCode}
Retrieve a full block from the blockchain
Usage: ./eosc get block [block\_num] [Options]
Usage: ./eosc get block [-j \{"block\_num\_or\_id":*\}] [OPTIONS]

Options:

  -n [ --block\_num ] arg  Block number
  -i [ --block\_id ] arg   Block id

  -h [ --help ]           Help screen
  -j [ --json ] arg       Json argument
  -v [ --received ]       Print received json
  -r [ --raw ]            Not pretty print
  -e [ --example ]        Usage example
\end{DoxyCode}
 
\begin{DoxyCode}
./eosc get block 25
##         block number: 25
##            timestamp: 2017-11-29T09:50:03
##     ref block prefix: 623236675
\end{DoxyCode}
 
\begin{DoxyCode}
./eosc get block 25 -v
\end{DoxyCode}
 
\begin{DoxyCode}
\{
    "previous": "00000018b5e0ffcd3dfede45bc261e3a04de9f1f40386a69821780e063a41448",
    "timestamp": "2017-11-29T09:50:03",
    "transaction\_merkle\_root": "0000000000000000000000000000000000000000000000000000000000000000",
    "producer": "initf",
    "producer\_changes": "",
    "producer\_signature":
       "2005db1a193cc3597fdc3bd38a4375df2a9f9593390f9431f7a9b53701cd46a1b5418b9cd68edbdf2127d6ececc4d66b7a190e72a97ce9adfcc750ef0a770f5619",
    "cycles": "",
    "id": "000000190857c9fb43d62525bd29dc321003789c075de593ce7224bde7fc2284",
    "block\_num": "25",
    "refBlockPrefix": "623236675"
\}
\end{DoxyCode}
 
\begin{DoxyCode}
./eosc get block 25 -v -r
\end{DoxyCode}
 
\begin{DoxyCode}

      \{"previous":"00000018b5e0ffcd3dfede45bc261e3a04de9f1f40386a69821780e063a41448","timestamp":"2017-11-29T09:50
      :03","transaction\_merkle\_root":"0000000000000000000000000000000000000000000000000000000000000000","producer"
      :"initf","producer\_changes":"","producer\_signature":"2005db1a193cc3597fdc3bd38a4375df2a9f9593390f9431f7a9b53
      701cd46a1b5418b9cd68edbdf2127d6ececc4d66b7a190e72a97ce9adfcc750ef0a770f5619","cycles":"","id":"000000190857c9fb43d62525bd29dc321003789c075de593ce7224bde7fc2284","block\_num":"25","refBlockPrefix":"623236675"\}
\end{DoxyCode}
 
\begin{DoxyCode}
./eosc get block -j '\{"block\_num\_or\_id":"56"\}'
##         block number: 56
##            timestamp: 2017-11-29T10:02:18
##     ref block prefix: 273573026
\end{DoxyCode}
 
\begin{DoxyCode}
./eosc get block --example
\end{DoxyCode}
 
\begin{DoxyCode}
Invoke 'get\_info' command:
get\_info get\_info;

\{
    "head\_block\_num": "9939",
    "last\_irreversible\_block\_num": "9924",
    "head\_block\_id": "000026d378f90b5d25dcf962fc44d637872218e5f826420a342f05a534d50bfc",
    "head\_block\_time": "2017-12-01T18:57:42",
    "head\_block\_producer": "initr",
    "recent\_slots": "0000000000000000000000000000000000000000000000000011111111111111",
    "participation\_rate": "0.21875000000000000"
\}


Use reference to the last block:
GetBlock GetBlock(
  get\_info.get<int>("last\_irreversible\_block\_num"));

\{
    "previous": "000026c35fb5d442be6d4e81a1347cce2c0184c4c2047d9e6dfc78b3bb325ac2",
    "timestamp": "2017-12-01T17:01:09",
    "transaction\_merkle\_root": "0000000000000000000000000000000000000000000000000000000000000000",
    "producer": "initn",
    "producer\_changes": "",
    "producer\_signature":
       "1f6984d14ee40ed9806ae14aa96531d874fc3417bf3f1b66c4b1d9c9402f3f90ef07c4523eb9a639ad632c181580aeb051385d718dc59ecc54d0f0e5de012b540f",
    "cycles": "",
    "id": "000026c44a2e8075a5b92813869bfb67b72b79ccb3f2e40ad815603c04d2fafd",
    "block\_num": "9924",
    "refBlockPrefix": "321436069"
\}
\end{DoxyCode}
 \subsection*{Library}

For us, real value is the library that runs the {\ttfamily tokenika eosc}, as we see the original eos library as not practical for our work. We need a light-\/weight thing, a cross-\/platform (good for windows) one.

Let you see a code snippet\+: 
\begin{DoxyCode}
#include <stdio.h>
#include <stdlib.h>
#include <iostream>
#include <string>

#include "EoscCommands/eosc\_get\_commands.hpp"

int main(int argc, char *argv[])
\{
  tokenika::eosc::get\_info get\_info; /* Call 'eosd' for 'get info'. */
  tokenika::eosc::GetBlock GetBlock( /* Call 'eosd' for 'get block', the last one. */
    get\_info.get<int>("last\_irreversible\_block\_num"));

  std::cout << GetBlock.toStringRcv() << std::endl;/* Print the response. */

  return 0;
\}
\end{DoxyCode}
 Here is the print-\/out\+: 
\begin{DoxyCode}
    "previous": "000028716589219b442afe9d140bc28eff4335aecd37d519b0105fca4c8e4a3f",
    "timestamp": "2017-12-01T19:18:27",
    "transaction\_merkle\_root": "0000000000000000000000000000000000000000000000000000000000000000",
    "producer": "inith",
    "producer\_changes": "",
    "producer\_signature":
       "1f510dec0bcd85847b7bead61f6deee7a5fb4108745e6ceaaa81804fe4700b561f7ca3f3f26f56fbfaf1e10fd3ba2999f8cbe165fd391b023334badcf894ba54dc",
    "cycles": "",
    "id": "00002872be99d0133ea104b42b771f3c7c2ea3736263dc9db3719728a2776976",
    "block\_num": "10354",
    "refBlockPrefix": "3020202302"
\}
\end{DoxyCode}


\label{_building}%
 \subsection*{\href{#toc}{\tt Building}}

\label{_dependencies}%
 \subsubsection*{\href{#toc}{\tt Dependencies}}

The only external dependency is the boost. We use the version 1\+\_\+65.

\label{_linux}%
 \subsubsection*{\href{#toc}{\tt Linux, Mac, ect.}}

C\+Make build. Starting in the installation directory\+:


\begin{DoxyCode}
mkdir build
cd build
cmake ..
make
\end{DoxyCode}


\label{_windows}%
 \subsubsection*{\href{#toc}{\tt Windows}}

There is an MS Visual Studio 17 solution in {\ttfamily eos\+\_\+visual\+\_\+studio} folder. You can start Visual Studio with file {\ttfamily eosc.\+sln} there, and you compile both the command library and `eosc\textquotesingle{} executable.

The VS solution has set both boost includes and libraries in relation to the {\ttfamily B\+O\+O\+S\+T\+\_\+\+R\+O\+OT} environmental variable\+: Configuration Properties $>$ V\+C++ Directories. Perhaps, you will have to adjust settings.

Now, the blockchain may be accessed from a Windows Command Prompt, if the {\ttfamily eosd} blockchain program is configured to be called from 